
\documentclass[numbers=noenddot, abstract=on]{scrreprt}

% Setting heading fonts to serif
\addtokomafont{disposition}{\rmfamily}

\usepackage[english]{babel} 
\usepackage[utf8]{inputenc}
\usepackage[backend=bibtex8]{biblatex}
\bibliography{bib/bibliography.bib}

% Using links in PDFs, but without the ugly borders
\usepackage{hyperref}
\hypersetup{
    colorlinks=false,
    pdfborder={0 0 0},
}


\begin{document}

\title{\bf E-Voting on Android Devices using the CGS97 Protocol}
\subject{Master Thesis Proposal}
\author{Juerg Ritter (\url{rittj1@bfh.ch})\\
\\
Bern University of Applied Sciences\\
Engineering and Information Technology\\
Biel, Switzerland\\}
\date{\today}
\publishers{Advisor: Prof. Dr. Rolf Haenni, Bern University of Applied
Sciences\\
Expert: Prof. Dr. X Y}
\maketitle




\begin{abstract}
The E-Voting group of the Bern University of Applied Sciences is planning to
evaluate distributed E-Voting systems for practical use. The term
``distributed'' in that context means that there is no central server
infrastructure involved in the voting process. The main application of such
E-Voting systems are polls with a low number of participants, for example the
board of directors in a company. The idea is, that the participants create an
ad-hoc network with mobile devices. The goal of this project is to implement a
distributed E-Voting system for Android devices. The communications schema among
the voting participants will rely on the proposal made by Cramer, Gennaro and
Schoenmakers in 1997 \cite{CGS97}. The cryptographic library UniCrypt and
InstaCircle which is an ad-hoc decentralized communication infrastructure are
other foundation blocks of this project. UniCrypt and InstaCircle are both
projects which have been implemented at the Bern University of Applied Sciences.
\end{abstract}

\tableofcontents

\chapter{Introduction}
One of the research fields of the Bern University of Applied Sciences is the
area of E-Voting. E-Voting has become a big field of research in the past couple
years. Still, there is no generic approach which meets all the criteria such as
privacy, transparency, etc. which we want in E-Voting. The E-Voting research
group of the Bern University of Applied Sciences \cite{www:EVG} tries to improve
this situation with the following approaches:
\begin{itemize}
  \item Develop new approaches and provide them to the community for review
  \item Take existing approaches and evaluate them in terms of practicability.
  These approaches are usually available as scientific papers
\end{itemize}

The evaluation of these approaches is usually done by implementing them into a
prototype level application to show that the approach actually works.

The E-Voting group would like to gain some experience on how distributed
E-Voting systems could be implemented and how they behave in practice. There are
some approaches which focus explicitely on distributed E-Voting systems, such as
the proposal of Khader et al. \cite{HKRS12}. In this project, we would like to
adapt the protocol proposed by Cramer et. al. \cite{CGS97} in a way that it can
be used as a distributed E-Voting system.

A possible use case of such a system could be an executive board or any other
committee that would like to vote on some matters without having their members
to reveal what exactly they voted for. The architecture of our system requires
that all the participants are in a confined space and are able to exchange some
sort of credential using a non-electronic channel.

The goal of this project is to build Systems which allows perform elections and
polls on mobile Android devices under the following terms:
\begin{itemize}
  \item Voting needs to be done ad-hoc, meaning that no infrastructure other
  than the participants mobile devices are required to perform an election or
  poll.
  \item The voting schema is based on the CGS97 approach proposed by Cramer et.
  al. \cite{CGS97}
\end{itemize}

In previous projects during the master studies, some groundwork has been
implemented which can now be used in this projects. The previously implemented
projects are the following:
\begin{itemize}
  \item \textbf{UniCrypt:} Unicrypt is a cryptographic library developed by the
  members of the E-Voting Group of the Bern University of Applied Sciences. It
  provides cryptographic building blocks such as the ElGamal crypto system, Zero
  Knowledge Proofs, digital Signatures, etc. 
  \item \textbf{InstaCircle: } InstaCircle provides a decentralized
  communication platform for Android devices. It allows to exchange messages
  using WiFi among a closed user group.
\end{itemize}
The projects mentioned above will be discussed in more detail in the section
Background and related work.

The time budget of this Master Thesis is one year, although the project will be
implemented part time. It is equivalent to 27 ECTS credits. 

\chapter{Background and related work}
This section gives an overview of the theoretical foundations and the work which
has been done previously on which this project is built on.

\section{The voting schema CGS97}
In 1997, Cramer, Gennaro and Schoenmakers proposed a scheme which allows to
do E-Voting in a secure and verifyable manner. 

\section{UniCrypt}

\section{InstaCircle}

\printbibliography

\end{document}
