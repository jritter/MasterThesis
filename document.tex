
\documentclass[numbers=noenddot, abstract=on]{scrreprt}

% Setting heading fonts to serif
\addtokomafont{disposition}{\rmfamily}

\usepackage[english]{babel} 
\usepackage[utf8]{inputenc}
\usepackage[backend=bibtex8]{biblatex}
\bibliography{bib/bibliography.bib}

% Using links in PDFs, but without the ugly borders
\usepackage{hyperref}
\hypersetup{
    colorlinks=false,
    pdfborder={0 0 0},
}

% we need these packages for the gantt chart
\usepackage{pgfgantt}
\usepackage{rotating}

% We need this to display the group symbols correctly
\usepackage{amssymb}
\usepackage{amsmath}

\begin{document}

\title{\bf E-Voting on Android Devices using the CGS97 Protocol}
\subject{Master Thesis Proposal}
\author{Juerg Ritter (\url{rittj1@bfh.ch})\\
\\
Bern University of Applied Sciences\\
Engineering and Information Technology\\
Biel, Switzerland\\}
\date{\today}
\publishers{Advisor: Prof. Dr. Rolf Haenni, Bern University of Applied
Sciences\\
Expert: Stephan Neumann, Technical University of Darmstadt}
\maketitle




\begin{abstract}
The E-Voting group of the Bern University of Applied Sciences is planning to
evaluate distributed E-Voting systems for practical use. The term
``distributed'' in that context means that there is no central server
infrastructure involved in the voting process. The main application of such
E-Voting systems are polls with a low number of participants, for example the
board of directors in a company. The idea is, that the participants create an
ad-hoc network with mobile devices. The goal of this project is to implement a
distributed E-Voting system for Android devices. The communications schema among
the voting participants will rely on the proposal made by Cramer, Gennaro and
Schoenmakers in 1997 \cite{CGS97}. The cryptographic library UniCrypt and
InstaCircle which is an ad-hoc decentralized communication infrastructure are
other foundation blocks of this project. UniCrypt and InstaCircle are both
projects which have been implemented at the Bern University of Applied Sciences.
\end{abstract}

\tableofcontents

\chapter{Introduction}
\label{cha:introduction}
One of the research fields of the Bern University of Applied Sciences is the
area of E-Voting. E-Voting has become a big field of research in the past couple
years. Still, there is no generic approach which meets all the criteria such as
privacy, transparency, etc. which we want in E-Voting. The E-Voting research
group of the Bern University of Applied Sciences \cite{www:EVG} tries to improve
this situation with the following approaches:
\begin{itemize}
  \item Develop new approaches and provide them to the community for review
  \item Take existing approaches and evaluate them in terms of practicability.
  These approaches are usually available as scientific papers
\end{itemize}

The evaluation of these approaches is usually done by implementing them into a
prototype level application to show that the approach actually works.

The E-Voting group would like to gain some experience on how distributed
E-Voting systems could be implemented and how they behave in practice. There are
some approaches which focus explicitely on distributed E-Voting systems, such as
the proposal of Khader et al. \cite{HKRS12}. In this project, we would like to
adapt the protocol proposed by Cramer et. al. \cite{CGS97} in a way that it can
be used as a distributed E-Voting system.

A possible use case of such a system could be an executive board or any other
committee that would like to vote on some matters without having their members
to reveal what exactly they voted for. The architecture of our system requires
that all the participants are in a confined space and are able to exchange some
sort of credential using a non-electronic channel.

The goal of this project is to build Systems which allows perform elections and
polls on mobile Android devices under the following terms:
\begin{itemize}
  \item Voting needs to be done ad-hoc, meaning that no infrastructure other
  than the participants mobile devices are required to perform an election or
  poll.
  \item The voting schema is based on the CGS97 approach proposed by Cramer et
  al.~\cite{CGS97}.
\end{itemize}

In previous projects during the master studies, some groundwork has been
implemented which can now be used as a foundation for this project. The
previously implemented projects are the following:
\begin{itemize}
  \item \textbf{UniCrypt:} Unicrypt is a cryptographic library developed by the
  members of the E-Voting Group of the Bern University of Applied Sciences. It
  provides cryptographic building blocks such as the ElGamal crypto system, Zero
  Knowledge Proofs, digital Signatures, etc. 
  \item \textbf{InstaCircle: } InstaCircle provides a decentralized
  communication platform for Android devices. It allows to exchange messages
  using WiFi among a closed user group.
\end{itemize}
The projects mentioned above will be discussed in more detail in the section
Background and related work.

The time budget of this Master Thesis is one year, although the project will be
implemented part time. It is equivalent to 27 ECTS credits. 

\chapter{Background and related work}
\label{cha:brw}
This section gives an overview of the theoretical foundations and the work which
has been done previously on which this project is built on.

\section{The voting schema CGS97}
\label{sec:CGS97}
In 1997, Cramer, Gennaro and Schoenmakers proposed a scheme which allows to
do E-Voting in a secure and verifyable manner \cite{CGS97}. The participants of
the protocol can be divided into four different categories:
\begin{itemize}
  \item \textbf{Voter: } A voter is somebody who is eligible to participate on
  an election or a poll. 
  \item \textbf{Trustee: } A trustee is somebody who helps setting up the
  election by creating an asymmetric keypair in cooperation with other trustees.
  At the end of the election phase, the trustees have to cooperate in order to
  reveal the result of the election or the poll.
  \item \textbf{Observer: } An observer is somebody who wants to verify that all
  the participants of an election or a poll behave as they are supposed to.
\end{itemize}

In the scenario of an ad-hoc voting system as we are going to develop during
this project, the three roles can be combined, meaning that all voters are also
acting as a trustee. 

\subsection{Cryptographic Building Blocks}
\label{sec:buildingblocks}
The CGS97 voting scheme is assembled from some well known cryptographic building
blocks which are shortly explained in this section.

\subsubsection{ElGamal Cryptosystem}
\label{sec:elgamal}
The ElGamal cryptosystem \cite{EG84},
proposed by Taher El Gamal in 1984, is the asymmetric cryptosystem which is mostly used in the
context of E-Voting. An asymmetric cryptosystem uses two keys to operate, one
which is used to encrypt a certain message (the public key) and another to
decrypt the message (the secret key). The security of this cryptosystem is based
on the fact that it is hard to compute the logarithm in discrete modular groups
and hence making the exponentiation in modular groups a one way function. 

In order to define an ElGamal cryptosystem, three parameters are required. Let
$p$ and $q$ be large prime numbers such that $q|p-1$. $p$ and $q$ are so
called save primes. $q$ defines the order of a subgroup $G_q$ of the
multiplicative modular group $\mathbb{Z}^*_p$. The last parameter needed is a random generator
$g$ of the group $G_q$. We can now derrive the asymmetric keypair containing the
secret key $x \in_R \mathbb{Z}_q$ and the public key $y=g^x \in G_q$. A message
$m \in G_q$ can be encrypted by first choosing a random value $r \in_R
\mathbb{Z}_q$, and then calculating $(a, b)=Enc_y(m, r)=(g^r, y^r \cdot m)$. The
tuple $(a, b)$ is the ciphertext of the message $m$. $m$ can be recovered using the secret key $x$
by calculating $m=Dec_x(a, b)=a^{-x} \cdot b$.

\subsubsection{Homomorphic encryption}
\label{sec:homenc}
The ElGamal cryptosystem has a property
which is important for building verifiable E-Voting schemes, namely the property
of homomorphism. Given two mathematical groups $(X,\oplus)$ and $(Y,\otimes)$, a
mathematical function $f:X \rightarrow Y$ is $(\oplus, \otimes)$-homomorphic if
the following condition holds:
\begin{equation}
f(m_1) \otimes f(m_2) = f(m_1 \oplus m_2) \notag
\end{equation}
The ElGamal function offers exactly that property with respect to the
multiplication operation for its encryption function $Enc_y:G_q \rightarrow
G_q$ using the same public key $y$:
\begin{equation}
Enc_y(m_1) \cdot Enc_y(m_2) = Enc_y(m_1 \cdot m_2) \notag
\end{equation}
This property allows to calculate the encrypted product of all messages out of
the encrypted messages and decrypt only the result. Applied to E-Voting, the
result can be calulated out of the encrypted ballots and only the final result
needs to be decrypted. The ballots themselves can remain encrypted, which is
important for maintaining privacy. The ElGamal cryptosystem is homomorphic
respective to multiplication, which is not very fortunate for counting votes.
Votes should be sumed up in order to get the final result. In order to reach
this, we use the following mathematical property:
\begin{equation}
x^a \cdot x^b = x^{a+b} \notag
\end{equation}

Applied to ElGamal, this transforms to the following equation:

\begin{equation}
Enc_y(g^{m_1}) \cdot Enc_y(g^{m_2}) = Enc_y(g^{m_1 + m_2}) \notag
\end{equation}

In order to obtain the sum of $m_1$ and $m_2$, the computation of a discrete
logarithm is required, which is considered as a hard problem. In the scenario of
E-Voting, we know what the possible solutions are. The number of possible
solution is merely the number of ballots which have been cast, so we can just
iterate through all the possible solutions and stop as soon as it matches with
the decrypted product.

\subsubsection{Secret sharing}
\label{sec:secretsharing}
In E-Voting scenarios it is crucial that not a
single entity can manipulate the result or reveal single votes. This
responsibility, or in our case the secret key which is needed to obtain the
final result, has to be spread across a set of trustees. In the CGS97 scheme,
this property is achieved by using a secret sharing mechanism as proposed by Adi
Shamir in 1979 \cite{SHA79}. This schema even allows to define a so called
\textit{threshold}, which defines the minimal amount of particpating trustees in
order to decrypt the result. Such a system is also known as a
$(t-n)$-threshold scheme, where $n$ defines the number of shares which are
issued at the beginning and $t$ defines the number of participants needed to
recover the secret. In a first step, a trusted dealer defines a polynomial
function $f(x)$ with degree $t-1$ and random coefficients. Each trustee $A_{i |
1 \leq i \leq n}$ gets its share $s_i=f(i)$. Secret itself, at this stage only
known by the trusted dealer, is defined by $s=f(0)$. In order to reproduce the
secret using the shares distributed among all trustees, we can interpolate the
shares in order to reproduce the coefficients and therefore also $f(x)$. Since a
polynomial function of degree $t-1$ needs at least $t$ points to reproduce using
an interpolation technique such as Lagrange interpolation, the secret can only
be reproduced if at least $t$ trustees are collaborating. 

The approach of Shamir is quite simple, has one major drawback though: It
requires a trusted authority which has the knowledge of the shared key. It it
would be nice to have a schema where the group of trustees collaborate to create
the shares in a way that nobody can derrive the private key unless a sufficient
amount of trusees collaborate. In 1991, Torben Pedersen proposed a schema where
this trusted dealer is no longer needed \cite{PED91}. In this schema, all the
trustees perform the steps of the trusted dealer in the schema of Shamir, namely
they generate a polynomial function with random coefficients of degree $t-1$ and
caluclate a point for each trustee in the system. These points are then
\textit{secretly} communicated individually to each trustee.

\subsubsection{Bulletin Board}
\label{sec:bulletinboard}
The so called bulletin board is the
public communication channel which is used to communicate between the
participants of the election or poll. It is a transscript of all the
communication steps between the participants and therefore contains encrypted
ballots, zero knowledge proofs, etc. The bulletin board is also available for
observers. Using the content of the bulletin board, everybody can verify that
all the participants are following the protocol or that dishonest participants
are excluded from the voting process. A voter can verify that his/her own
ballot is counted properly and also reflects in the final result. In theory, it
is not possible to delete anything from a bulletin board (append only). Since
the bulletin board is a good target for a denial-of-service attack, it is a good
idea to replicate the content of the bulletin board to multiple systems.

\section{UniCrypt}
\label{sec:unicrypt}
UniCrypt is the name of a cryptographic library which has been developed by the
E-Voting group of the Bern University of Applied Sciences. The main goal of this
project was to create a plattform on which upcoming projects can be built on.
UniCrypt tries to look at cryptographic functions such as crypto systems,
signature systems, hash functions etc. from a mathematical point of view. Cipher
text, plain text, signatures, etc. are treated as elements of mathematical
groups, the cryptographic functions are treated as mathematical functions. This
is required because in secure protocols as we use in the area of E-Voting, we
depend on the plain textbook implementation of the cryptographic functions
without functions like automatic padding, encoding, etc. This is the main
difference between UniCrypt and other cryptographic libraries.

At the moment, UniCrypt has only been used in classic Java environments, but not
on mobile devices. Futher information about UniCrypt can be found in
\cite{ritter12}.

\section{InstaCircle}
\label{sec:instacircle}
InstaCircle is the name of a project that has been implemented as a preparation
of this master thesis. It is intended as a platform to connect mobile devices
in a confined space to a Wi-Fi ad-hoc network in order to exchange broadcast
and unicast messages. In order to keep the network traffic as low as possible,
the basic communication relies on broadcast messages. Since it is not possible
implement a reliable protocol using broadcast techniques, mechanisms to
compensate message losses have been implemented. These resending mechanisms rely
on a reliable unicast channel.

From a usability perspective, the process on agreeing to a communication channel
for an ad-hoc network is quite challenging. All the participants need to have
the know-how to switch their devices to the correct Wi-Fi network and enter the
correct keys and passwords. InstaCircle tries to improve the usability by
allowing the users to exchange the configuration by sharing a QR code or a NFC
tag. Once a user has set up a conversation, other participants can join the
conversation by reading the QR code or the NFC tag shared by the initiator of
the conversation.

InstaCircle is currently available for Android devices only. Futher information
about InstaCircle can be found in \cite{ritter13a}.

\chapter{Expected Results}
\label{cha:expectedresults}
The final result of this project is a fully-fledged distributed E-Voting
application which runs on Android devices and does not require any equipment
other than the Android smartphones or tablets. Since a considerable amount of
people carry such devices, which allows setting up spontaneous polls. The
implementation can be split up in several packages, which are shortly explained
in the following section.

\section{Threshold Secret Sharing}
\label{sec:thresholdsecretsharing}
In order to ensure privacy of the ballots, the CGS97 \cite{CGS97} E-Voting
protocol uses a threshold secret sharing mechanism which splits up the secret
key which will later be used for decrypting the result among an certain number
of trustees. Since the protocol allows participants to impersonate multiple
rules, all participants could also be trustees. In the scenario of an ad-hoc
E-Voting setup, it certainly makes sense to assign the role of the trustee to
all participants. This is because of the following reasons:
\begin{itemize}
  \item In many cases, the number of participants in an ad-hoc poll can be so
  small that it would be hard to find a trustee subset
  \item In classical large-scale elections or polls, the role of the trustee
  will probably be assigned to organizations such as political parties,
  governmental authorities, election observation organizations, etc. In
  spontaneous ad-hoc polls, the question of who gets to be a trustee is not an
  obvious one, so it is probably the best choice to assign this role to all
  participants.
  \item Assigning the role of a trustee to all participants other than just to a
  subset of participants means that the handling of the application would be the
  same for all the participants (except for one special participant who defines
  the question and the allowed options of the poll). That simplifies the design
  of the User interface and avoids confusion about who has to do what during the
  poll.
\end{itemize}

The CGS97 protocol suggests to use threshold secret sharing schema due to Torben
Pedersen \cite{PED91}, which adds a distributed key generation mechanism to the
schema proposed by Adi Shamir \cite{SHA79}. Especially the key generation part
of the protocol involves a lot of steps and interaction between participants.

The secret sharing mechanism is currently not part of the InstaCircle library
and has to be implemented. The implementation such a threshold mechanism in
InstaCircle is a goal for this project.

\section{Graphical User Interface}
\label{sec:gui}
In many cases, cryptographic functions which are heavily used in the context of
E-Voting have a negative impact on the usability. Many steps such as key
generation, decryption, etc. have to be done in order to guarantee a poll which
can't be compromised, and these steps bring a certain complexity into the
handling of the application. Therefore it is crucial to guide the user
through the process, and this can only be done by providing them a carefully
crafted user interface. We can divide the required user interface into the
following components:
\begin{itemize}
  \item \textbf{Administration:} The very first step in a poll would be to
  define the question of the poll and the allowed options. Also, some sort of
  access credential has to be defined. This step has to be performed by one
  special participant, let's define this special participant as the
  \textit{initiator} of the poll. Functionality in order to open and close the
  voting period is also necessary in this part.
  \item \textbf{Voting:} One part of the interface has to deal with the
  functionality of casting a vote. It basically displays the question of the
  poll and provides functionality to choose one of the available options.
  \item \textbf{Bulletin board:} One building block of the CGS97 protocol
  \cite{CGS97} is the so called bulletin board where the whole transscript of
  messages between the participants is saved. All these information needs to be displayed on the
  user interface so that the participant has the possibility to verify the
  validity of the whole E-Voting process (see section \ref{sec:bulletinboard}
  on page \pageref{sec:bulletinboard}).
\end{itemize}

\section{E-Voting Functionality}
\label{sec:evotingfunctionality}
The main challenge of this project is the implementation of the logic of CGS97 
\cite{CGS97}, namely assemble the cryptographic building blocks to a working
application and tie it to the user interface and to InstaCircle for communicating to the other
participants. This part is heavily based on already existing components, namely
InstaCircle which provides the decentralized communication infrastructure and
UniCrypt which provides the implementation of the cryptographic building blocks
which are used in CGS97.

\section{Planning}
\begin{sidewaysfigure}[ph]
\centering
\begin{ganttchart}
[vgrid, hgrid,
group/.style={draw=black, fill=black!50}, y unit chart=1cm]{26}
\gantttitle{Planning Master Thesis (part 1)}{26} \\
\gantttitlelist{8,...,33}{1} \\
\ganttmilestone{Kickoff}{0.5} \\
\ganttgroup{Writing of Project Proposal}{1}{4} \\
\ganttgroup{Secret Sharing scheme}{4}{8} \\
\ganttbar{Analyzing the possibilities of threshold secret sharing}{4}{6} \\
\ganttbar{Implementing secret sharing into UniCrypt}{5}{8} \\
\ganttbar{Testing the secret sharing scheme}{7}{8} \\
\ganttgroup{Evalutation of UniCrypt on Android}{9}{10} \\
\ganttbar{Testing out Example code on Android}{9}{9} \\
\ganttbar{Integration of code into Android project}{10}{10} \\
\ganttgroup{Implementation of E-Voting Application of Android}{11}{26} \\
\ganttbar{Design of storyboard}{11}{12}\\
\ganttbar{Implementation of the User Interface}{12}{15}\\
\ganttbar{Implementation of the E-Voting logic}{20}{26}\\
\ganttgroup{Testing}{30}{30} \\
\ganttgroup{Documentation / Report}{1}{8} \\
\ganttmilestone{Handin}{30}
\end{ganttchart}
\caption{Planning part 1}
\label{fig:planningpart1}
\end{sidewaysfigure}

\begin{sidewaysfigure}[ph]
\centering
\begin{ganttchart}
[vgrid, hgrid,
group/.style={draw=black, fill=black!50}, y unit chart=1cm]{25}
\gantttitle{Planning Master Thesis (part 2)}{25} \\
\gantttitlelist{34,35,36,37,38,39,40,41,42,43,44,45,46,47,48,49,50,51,52,1,2,3,4,5,6}{1} \\
\ganttmilestone{Kickoff}{100} \\
\ganttgroup{Writing of Project Proposal}{100}{100} \\
\ganttgroup{Secret Sharing scheme}{100}{100} \\
\ganttbar{Analyzing the possibilities of threshold secret sharing}{100}{100} \\
\ganttbar{Implementing secret sharing into UniCrypt}{100}{100} \\
\ganttbar{Testing the secret sharing scheme}{100}{100} \\
\ganttgroup{Evalutation of UniCrypt on Android}{100}{100} \\
\ganttbar{Testing out Example code on Android}{100}{100} \\
\ganttbar{Integration of code into Android project}{100}{100} \\
\ganttgroup{Implementation of E-Voting Application of Android}{1}{13} \\
\ganttbar{Design of storyboard}{100}{100}\\
\ganttbar{Implementation of the User Interface}{100}{100}\\
\ganttbar{Implementation of the E-Voting logic}{1}{13}\\
\ganttgroup{Testing}{7}{19} \\
\ganttgroup{Documentation / Report}{14}{24.5} \\
\ganttmilestone{Handin}{24.5}
\end{ganttchart}
\caption{Planning part 2}
\label{fig:planningpart2}
\end{sidewaysfigure}

\printbibliography

\end{document}
