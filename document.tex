
\documentclass[numbers=noenddot, abstract=on]{scrreprt}

% Setting heading fonts to serif
\addtokomafont{disposition}{\rmfamily}

\usepackage[english]{babel} 
\usepackage[utf8]{inputenc}
\usepackage[backend=bibtex8]{biblatex}
\bibliography{bib/bibliography.bib}

% Using links in PDFs, but without the ugly borders
\usepackage{hyperref}
\hypersetup{
    colorlinks=false,
    pdfborder={0 0 0},
}


\begin{document}

\title{\bf E-Voting on Android Devices using the CGS97 Protocol}
\subject{Master Thesis Proposal}
\author{Juerg Ritter (\url{rittj1@bfh.ch})\\
\\
Bern University of Applied Sciences\\
Engineering and Information Technology\\
Biel, Switzerland\\}
\date{\today}
\publishers{Advisor: Prof. Dr. Rolf Haenni, Bern University of Applied
Sciences\\
Expert: Stephan Neumann, Technical University of Darmstadt}
\maketitle




\begin{abstract}
The E-Voting group of the Bern University of Applied Sciences is planning to
evaluate distributed E-Voting systems for practical use. The term
``distributed'' in that context means that there is no central server
infrastructure involved in the voting process. The main application of such
E-Voting systems are polls with a low number of participants, for example the
board of directors in a company. The idea is, that the participants create an
ad-hoc network with mobile devices. The goal of this project is to implement a
distributed E-Voting system for Android devices. The communications schema among
the voting participants will rely on the proposal made by Cramer, Gennaro and
Schoenmakers in 1997 \cite{CGS97}. The cryptographic library UniCrypt and
InstaCircle which is an ad-hoc decentralized communication infrastructure are
other foundation blocks of this project. UniCrypt and InstaCircle are both
projects which have been implemented at the Bern University of Applied Sciences.
\end{abstract}

\tableofcontents

\chapter{Introduction}
One of the research fields of the Bern University of Applied Sciences is the
area of E-Voting. E-Voting has become a big field of research in the past couple
years. Still, there is no generic approach which meets all the criteria such as
privacy, transparency, etc. which we want in E-Voting. The E-Voting research
group of the Bern University of Applied Sciences \cite{www:EVG} tries to improve
this situation with the following approaches:
\begin{itemize}
  \item Develop new approaches and provide them to the community for review
  \item Take existing approaches and evaluate them in terms of practicability.
  These approaches are usually available as scientific papers
\end{itemize}

The evaluation of these approaches is usually done by implementing them into a
prototype level application to show that the approach actually works.

The E-Voting group would like to gain some experience on how distributed
E-Voting systems could be implemented and how they behave in practice. There are
some approaches which focus explicitely on distributed E-Voting systems, such as
the proposal of Khader et al. \cite{HKRS12}. In this project, we would like to
adapt the protocol proposed by Cramer et. al. \cite{CGS97} in a way that it can
be used as a distributed E-Voting system.

A possible use case of such a system could be an executive board or any other
committee that would like to vote on some matters without having their members
to reveal what exactly they voted for. The architecture of our system requires
that all the participants are in a confined space and are able to exchange some
sort of credential using a non-electronic channel.

The goal of this project is to build Systems which allows perform elections and
polls on mobile Android devices under the following terms:
\begin{itemize}
  \item Voting needs to be done ad-hoc, meaning that no infrastructure other
  than the participants mobile devices are required to perform an election or
  poll.
  \item The voting schema is based on the CGS97 approach proposed by Cramer et.
  al. \cite{CGS97}
\end{itemize}

In previous projects during the master studies, some groundwork has been
implemented which can now be used as a foundation for this project. The
previously implemented projects are the following:
\begin{itemize}
  \item \textbf{UniCrypt:} Unicrypt is a cryptographic library developed by the
  members of the E-Voting Group of the Bern University of Applied Sciences. It
  provides cryptographic building blocks such as the ElGamal crypto system, Zero
  Knowledge Proofs, digital Signatures, etc. 
  \item \textbf{InstaCircle: } InstaCircle provides a decentralized
  communication platform for Android devices. It allows to exchange messages
  using WiFi among a closed user group.
\end{itemize}
The projects mentioned above will be discussed in more detail in the section
Background and related work.

The time budget of this Master Thesis is one year, although the project will be
implemented part time. It is equivalent to 27 ECTS credits. 

\chapter{Background and related work}
This section gives an overview of the theoretical foundations and the work which
has been done previously on which this project is built on.

\section{The voting schema CGS97}
In 1997, Cramer, Gennaro and Schoenmakers proposed a scheme which allows to
do E-Voting in a secure and verifyable manner \cite{CGS97}. The participants of
the protocol can be divided into four different categories:
\begin{itemize}
  \item \textbf{Voter: } A voter is somebody who is eligible to participate on
  an election or a poll. 
  \item \textbf{Trustee: } A trustee is somebody who helps setting up the
  election by creating an asymmetric keypair in cooperation with other trustees.
  At the end of the election phase, the trustees have to cooperate in order to
  reveal the result of the election or the poll.
  \item \textbf{Observer: } An observer is somebody who wants to verify that all
  the participants of an election or a poll behave as they are supposed to.
\end{itemize}

In the scenario of an ad-hoc voting system as we are going to develop during
this project, the three roles can be combined, meaning that all voters are also
acting as a trustee. 

\subsection{Cryptographic Building Blocks}
The CGS97 voting scheme is assembled from some well known cryptographic building
blocks which are shortly explained in this section.

\paragraph{ElGamal Cryptosystem.} The ElGamal cryptosystem \cite{EG84}, proposed
by Taher El Gamal in 1984, is the asymmetric cryptosystem which is mostly used in the
context of E-Voting. An asymmetric cryptosystem uses two keys to operate, one
which is used to encrypt a certain message (the public key) and another to
decrypt the message (the secret key). The security of this cryptosystem is based
on the fact that it is hard to compute the logarithm in discrete modular groups
and hence making the exponentiation in modular groups a one way function. 

\paragraph{Secret sharing.} In E-Voting scenarios it is crucial that not a
single entity can manipulate the result or reveal single votes. This
responsibility, or in our case the secret key which is needed to obtain the
final result, has to be spread across a set of trustees. In the CGS97 scheme,
this property is achieved by using a secret sharing mechanism as proposed by Adi
Shamir in 1997 \cite{SHA79}. This schema even allows to define a so called
\textit{threshold}, which defines the minimal amount of particpating trustees in
order to decrypt the result. Such a system is also known as a
$(t-n)$-threshold scheme, where $n$ defines the number of shares which are
issued at the beginning and $t$ defines the number of participants needed to
recover the secret. In a first step, a trusted dealer defines a polynomial
function $f(x)$ with degree $t-1$ and random coefficients. Each trustee $A_{i |
1 \leq i \leq n}$ gets its share $s_i=f(i)$. Secret itself, at this stage only
known by the trusted dealer, is defined by $s=f(0)$. In order to reproduce the
secret using the shares distributed among all trustees, we can interpolate the
shares in order to reproduce the coefficients and therefore also $f(x)$. Since a
polynomial function of degree $t-1$ needs at least $t$ points to reproduce using
an interpolation technique such as Lagrange interpolation, the secret can only
be reproduced if at least $t$ trustees are collaborating. 

\section{UniCrypt}
UniCrypt is the name of a cryptographic library which has been developed by the
E-Voting group of the Bern University of Applied Sciences.

\section{InstaCircle}

\printbibliography

\end{document}
