
\documentclass[numbers=noenddot, abstract=on, a4paper, headsepline,
footsepline, oneside, draft=off]{scrreprt}

% Setting heading fonts to serif
\addtokomafont{disposition}{\rmfamily}

% Setting description label fonts to serif
\addtokomafont{descriptionlabel}{\rmfamily}


\usepackage[english]{babel} 
\usepackage[utf8]{inputenc}

% biblatex
\usepackage[backend=bibtex8]{biblatex}
\bibliography{bib/bibliography.bib}

% switch last names in bibliography to small caps
\renewcommand*{\mkbibnamelast}[1]{\textsc{#1}}

% we need these packages for the gantt chart
\usepackage{pgfgantt}
\usepackage{rotating}

% We need this to display the group symbols correctly
\usepackage{amssymb}
\usepackage{amsmath}

\usepackage{upgreek}

% activate the headings
\pagestyle{headings}
\setkomafont{pagehead}{\scshape}

% UML stuff
\usepackage{tikz}
\usepackage{tikz-uml}

\usetikzlibrary{matrix, arrows}

% Intelligent references
\usepackage{varioref} 


% Draft watermark
\usepackage[firstpage]{draftwatermark}
\SetWatermarkScale{5}
\SetWatermarkLightness{0.9}



% improve word spacing
\usepackage{microtype}

% Using links in PDFs, but without the ugly borders
\usepackage{hyperref}
\hypersetup{
    colorlinks=false,
    pdfborder={0 0 0},
    pdfauthor={Juerg Ritter},
    pdftitle={Decentralized E-Voting on Android Devices Using Homomorphic
    Tallying},
    pdfsubject={Master Thesis},
    pdfkeywords={E-Voting;Android;Master Thesis},
}

\usepackage{cleveref}


% definitions of own commands
%\newcommand{\myref}[1]{(see section \ref{#1} on page \pageref{#1})}
\newcommand{\myref}[1]{(see \Vref{#1})}

\begin{document}

\title{\bf Decentralized E-Voting on Android Devices Using Homomorphic Tallying}
\subject{Master Thesis}
\author{Juerg Ritter (\url{rittj1@bfh.ch})\\
\\
Bern University of Applied Sciences\\
Engineering and Information Technology\\
CH-2501 Biel, Switzerland\\
}
\date{\today}
\publishers{Advisor:\\
Prof. Dr. Rolf Haenni, Bern University of Applied Sciences\\
\bigskip
Expert:\\
Stephan Neumann, Technical University of Darmstadt}
\maketitle


\begin{abstract}
The E-Voting Group of the Bern University of Applied Sciences is planning to
evaluate decentralized e-voting systems for practical use. The term
``decentralized'' in that context means that there is no central server
infrastructure involved in the voting process. The main application of such
e-voting systems are polls with a low number of participants, for example the
board of directors in a company. The idea is, that the participants create an
ad-hoc network with mobile devices. The main goal of this master thesis is to
implement a decentralized e-voting system for Android based mobile devices. We
are using a voting scheme that uses homomorphic tallying and secret sharing to
guarantee privacy and verifiability. It also provides robustness in the sense
that it is not possible for a single party to break the voting process just by
refusing the collaboration at a certain step throughout the voting process. A
big focus area is also the usability of this system, it should be possible for
participants without extensive knowledge in the area of information security to
use this e-voting system.
\end{abstract}

\tableofcontents

\chapter{Introduction}
\label{cha:introduction}
The history of voting on a particular question in order determine the will of a
group of people goes far back to the ancient Greeks, who laid the groundwork of
today's democratic societies. Over the years, the purpose of holding elections
remained the same, while the procedures on how this is done has changed
significantly. The most ancient way of expressing the will by a hand sign has
been replaced by modern technologies. The 20th century has
seen a lot of changes in this area. Paper based voting has been introduced, and with the
industrialization the first machines appeared which should help counting
the votes appeared on the surface. In today's internet society where a lot of
tasks such as banking, shopping, mailing, etc. has been revolutionized by the
internet, it is not surprising that there are a lot of efforts going on to
revolutionize the process of voting as well.

One of the research fields of the Bern University of Applied Sciences is the
area of e-voting. E-voting has become a big field of research in the past couple
years. Still, there is no generic approach which meets all the criteria such as
privacy, transparency, etc. which we want in e-voting. The e-voting research
group of the Bern University of Applied Sciences \cite{www:EVG} tries to improve
this situation with the following approaches:
\begin{itemize}
  \item Develop new approaches and provide them to the community for review.
  \item Take existing approaches and evaluate them in terms of practicability.
  These approaches are usually available as scientific papers.
\end{itemize}

The evaluation of these approaches is usually done by implementing them into a
prototype level application to show that the approach actually works. The
E-Voting Group would like to gain some experience on how decentralized e-voting
systems could be implemented and how they behave in practice. There are some
approaches which focus explicitly on decentralized e-voting systems, such as the
proposal of Khader et al. \cite{HKRS12}. In this project, we would like to adapt
the protocol proposed by Cramer et. al. \cite{CGS97} in a way that it can be
used as a decentralized e-voting system.

A possible use case of such a system could be an executive board or any other
committee that would like to vote on some matters without having their members
to reveal what exactly they voted for. The architecture of our system requires
that all the participants are in a confined space and are able to exchange some
sort of credential using a communication channel which relies on physical
proximity (e.g. Near Field Communication) or visual contact (e.g. QR Codes).

The goal of this project is to build systems which allows perform elections and
polls on mobile Android devices under the following terms:
\begin{itemize}
  \item Voting needs to be done ad-hoc, meaning that no infrastructure other
  than the participants' mobile devices are required to perform an election or
  poll.
  \item The voting scheme is based on the CGS97 approach proposed by Cramer et
  al.~\cite{CGS97}.
  \item It must be possible to perform elections or polls with the type \emph{one-out-of-n},
  meaning the voter can choose exactly one option out of $n$ possible options.
  It will not be possible to perform elections where multiple candidates can be
  elected for example.
\end{itemize}

In previous projects during the master studies, some groundwork has been
implemented which can now be used as a foundation for this project. The
previously implemented projects are the following:
\begin{itemize}
  \item \textbf{UniCrypt:} Unicrypt is a cryptographic library developed by the
  members of the E-Voting Group of the Bern University of Applied Sciences. It
  provides cryptographic building blocks such as the ElGamal cryptosystem, zero
  knowledge proofs, digital signatures, etc. 
  \item \textbf{InstaCircle: } InstaCircle provides a decentralized
  communication platform for Android devices. It allows to exchange messages
  using Wi-Fi among a closed user group.
\end{itemize}
The projects mentioned above will be discussed in more detail in the Chapter
``Background and Related Work'' \myref{cha:brw}.

The time budget of this master thesis is one year, although the project will be
implemented part time. It is equivalent to 27 ECTS credits.
\Vref{cha:expectedresults} gives a more in-depth specification about the results which are expected at the end of this master thesis.

\section{Contribution and Goals}
\label{cha:contributiongoals}
The final result of this project is a fully-fledged decentralized e-voting
application which runs on Android devices and does not require any equipment
other than the Android smartphones or tablets. Since a considerable amount of
people carry such devices, which allows setting up spontaneous polls. The
implementation can be split up in several packages, which are shortly explained
in the following Section.

\subsection{Implementation of a Threshold Secret Sharing scheme in UniCrypt}
\label{sec:thresholdsecretsharing}
In order to ensure privacy of the ballots, the CGS97 \cite{CGS97} e-voting
protocol uses a threshold secret sharing mechanism which splits up the secret
key which will later be used for decrypting the result among an certain number
of trustees. Since the protocol allows participants to impersonate multiple
rules, all participants could also be trustees. In the scenario of an ad-hoc
e-voting setup, it certainly makes sense to assign the role of the trustee to
all participants. This is because of the following reasons:
\begin{itemize}
  \item In many cases, the number of participants in an ad-hoc poll can be so
  small that it would be hard to find an even smaller subset of participants
  which will act as trustees.
  \item In classical large-scale elections or polls, the role of the trustee
  will probably be assigned to organizations such as political parties,
  governmental authorities, election observation organizations, etc. In
  spontaneous ad-hoc polls, the question of who gets to be a trustee is not an
  obvious one, so it is probably the best choice to assign this role to all
  participants.
  \item Assigning the role of a trustee to all participants other than just to a
  subset of participants means that the handling of the application would be the
  same for all the participants (except for one special participant who defines
  the question and the allowed options of the poll). That simplifies the design
  of the user interface and avoids confusion about who has to do what during the
  poll.
\end{itemize}

The CGS97 protocol suggests to use threshold secret sharing scheme due to Torben
Pryds Pedersen \cite{PED91}, which adds a distributed key generation mechanism to the
scheme proposed by Adi Shamir \cite{Shamir79}. 

We are considering replacing the secret sharing scheme with a variant that
requires less steps during the distributed key generation phase. So far this
scheme was only subject to some discussions and has to be analyzed and specified
in more detail. This work is also done as a part of this master thesis.

There is currently no threshold secret sharing mechanism implemented in
UniCrypt. The implementation such a mechanism in InstaCircle is a goal
for this project.


\subsection{Implementation of a Proof of Validity in UniCrypt}
\label{sec:proofofvalidity}
In \Vref{sec:zeroknowledgeproofs} we have seen three types of zero-knowledge
proofs which are required as a building block of CGS97. The InstaCircle library
currently implements two types, namely the proof of knowledge of discrete
logarithm and the proof of equality of discrete logarithm. What's currently
missing is the proof of validity, which allows to create a proof proving that
a given image of a homomorphic one-way function corresponds to a preimage which
belongs to a given set of allowed preimages. 

\subsection{Graphical User Interface}
\label{sec:gui}
In many cases, cryptographic functions which are heavily used in the context of
e-voting have a negative impact on the usability. Many steps such as key
generation, decryption, etc. have to be done in order to guarantee a poll which
can't be compromised, and these steps bring a certain complexity into the
handling of the application. Therefore it is crucial to guide the user
through the process, and this can only be done by providing them a carefully
crafted user interface. We can divide the required user interface into the
following components:
\begin{itemize}
  \item \textbf{Administration:} The very first step in a poll would be to
  define the question of the poll and the allowed options. Also, some sort of
  access credential has to be defined. This step has to be performed by one
  special participant, let's define this special participant as the
  \textit{initiator} of the poll. Functionality in order to open and close the
  voting period is also necessary in this part.
  \item \textbf{Voting:} One part of the interface has to deal with the
  functionality of casting a vote. It basically displays the question of the
  poll and provides functionality to choose one of the available options.
  \item \textbf{Bulletin board:} One building block of the CGS97 protocol
  \cite{CGS97} is the so called bulletin board where the whole transcript of
  messages between the participants is saved. All these information needs to be displayed on the
  user interface so that the participant has the possibility to verify the
  validity of the whole e-voting process \myref{sec:bulletinboard}.
\end{itemize}

\subsection{E-Voting Functionality}
\label{sec:evotingfunctionality}
The main challenge of this project is the implementation of the logic of CGS97 
\cite{CGS97}, namely assemble the cryptographic building blocks to a working
application and tie it to the user interface and to InstaCircle for communicating to the other
participants. This part is heavily based on already existing components, namely
InstaCircle which provides the decentralized communication infrastructure and
UniCrypt which provides the implementation of the cryptographic building blocks
which are used in CGS97 \myref{sec:buildingblocks}. The final product needs to
be able to handle a voting scenario \emph{one-out-of-n}, meaning that a user can
choose exactly one option out of $n$ options. This allows to set up polls which are more
complex than a simple \emph{Yes-No} poll. At the other end of the spectrum would
be the \emph{m-out-of-n} scenario, that means that $m$ options can be picked out
of $n$ options. The \emph{m-out-of-n} scenario is not a must for this project.

\section{Outline}
\Vref{cha:brw} gives an overview of the work on which this project has
been based on. The cryptographic building blocks which are used are discussed,
as well as the e-voting scheme on which this work is based on. We also discuss
projects which have been implemented as a preparation of this master thesis.
\Vref{cha:results} explains the resulting implementation in detail.

\chapter{Background and Related Work}
\label{cha:brw}
This section gives an overview of the theoretical foundations and some projects
which have been implemented earlier in order to serve as a foundation for this
project. 

\section{E-Voting in 2013}
\label{sec:evoting}
In today's internet world, many tasks such as sending mail, doing bank
transactions, booking flights etc. can be done using the internet. Instead of
going to the post office, to the bank or the travel agency, one can complete
these tasks in a very comfortable fashion at home in front of the computer. Not
surprisingly, people have thought about replace the way to the voting station
with an internet based solution. This so called e-voting has appeared on the
surface soon after the quick rise of the internet, but so far the big
breakthrough has not happened, which might be surprising at the first sight.
Instead, the area of e-voting has become a big research area, tightly related to
the area of cryptography. Let's have a closer look at the reasons why e-voting
couldn't find it's way to success so far. For a better understanding, we should
identify the properties that we expect from an e-voting system. Over the time,
some properties and their definitions have emerged.

\begin{description}
  \item[Democracy:] Only eligible voters can vote, and each voter can
  cast at most one ballot which will be included in the result.
  \item[Accuracy:] The result is derived only from \emph{all} valid
  votes as they were cast, i.e. cast votes can not be modified.
  \item[Universal verifiability:] Anyone can verify the correctness of
  the result.
  \item[Individual verifiability:] A voter can can verify that her vote
  is included in the result as she cast it.
  \item[Privacy:] Nobody can obtain information how a voter has
  voted.
  \item[Receipt-freeness:] The voter can not prove to anybody else how
  she voted.
  \item[Coercion-resistance:] Noone can force a voter to vote in a
  particular way, or prevent a voter to vote at all.
  \item[Robustness:] An attacker is not able to disrupt an election
  process or change the outcome by damaging a small set of system components. 
\end{description}

All these properties are discussed in detail in \cite{HS11} and \cite{Jonker09}.
As it turns out, it is not at all easy to build a system that meets all these
desired requirements. Not even the classic paper based voting meets all the
requirements, for example the verifiability is not given in a classic voting
scenario. Researchers are trying to develop schemes towards these requirements
using cryptographic building blocks. The robustness property is probably the
biggest factor that prevents the breakthrough of e-voting on large scales.
e-voting systems are the perfect targets for an intended manipulation or a
boycott of an election. That's why many governments use e-voting only on small
scales so that the electorate is small enough that it is statistically unlikely
to decide the outcome of the election.

\section{Cryptographic Building Blocks}
\label{sec:buildingblocks}
The CGS97 voting scheme is assembled from some well known cryptographic building
blocks which are shortly explained in this section.

\subsection{ElGamal Cryptosystem}
\label{sec:elgamal}
The ElGamal cryptosystem \cite{EG84}, proposed by Taher El Gamal in 1984, is the
asymmetric cryptosystem which is mostly used in the context of e-voting. An
asymmetric cryptosystem uses two keys to operate, one which is used to encrypt a
certain message (the public key) and another to decrypt the message (the secret
key). The security of this cryptosystem is based on the fact that it is hard to
compute the logarithm in discrete modular groups and hence making the
exponentiation in modular groups a one way function. A major advantage of the
ElGamal cryptosystem is the fact, that the encryption function includes a random
value. Especially in the context of e-voting this is a crucial property, because
when encrypting a value (for example the value $1$ which means \textit{Yes}),
the resulting ciphertext is always different. That's why ElGamal is also called
a \textit{randomized} cryptosystem.

In order to define an ElGamal cryptosystem, three parameters are required. Let
$p$ and $q$ be large prime numbers such that $q|p-1$\footnote{In the case of
$p=2q+1$ where $p$ and $q$ are primes, $p$ is called \emph{save prime} and $q$
is called \emph{Sophie Germain prime}.}.
$q$ defines the order of a subgroup $G_q$ of the multiplicative modular group
$\mathbb{Z}^*_p$. $\mathbb{Z}_q$ is an additive modular group of order $q$. The
last parameter needed is a random generator $g$ of the group $G_q$. Futher
information concerning group theory can be found in Chapter 2 of the Handbook of
Applied Cryptography by Alfred Menezes et. al. \cite{book:hac}. We can now
derive the asymmetric key pair containing the secret key $x \in_R \mathbb{Z}_q$ and the public key $y=g^x \in G_q$. A message $m \in G_q$ can be encrypted by first choosing a random value $r \in_R \mathbb{Z}_q$, and then applying the following function:
\begin{equation}
Enc_y : G_q \times \mathbb{Z}_q \rightarrow G_q \times G_q,
(a, b)=Enc_y(m, r)=(g^r, y^r \cdot m) \notag
\end{equation}
The tuple $(a, b)$ is the ciphertext of the message $m$. $m$ can be recovered using the secret key $x$
by applying the decryption function:
\begin{equation}
Dec_x : G_q \times G_q \rightarrow G_q,
m=Dec_x(a, b)=a^{-x} \cdot b \notag
\end{equation}

\subsubsection{Exponential ElGamal}
\label{sec:expelgamal}
In the area of e-voting, we often find a slight variation of ElGamal which is
called \emph{exponential ElGamal}. In exponential ElGamal, we encode a message
$\hat{m}$ by using a generator $\hat{g}$ of $G_q$ and raise it to
the power of the message $\hat{m} \in \mathbb{Z}_q$. In many cases, the
ElGamal parameter $g$ is used as $\hat{g}$, which makes it unnecessary to
define an fourth predefined parameter. Note that the message $\hat{m}$ is an
element of the additive group $\mathbb{Z}_q$ as opposed to the classical
ElGamal where $m$ is an element of the multiplicative group $G_q$. Note that
elements of the group $\mathbb{Z}_q$ are much easier to find as all integers
between $0$ and $p-1$ belong to this group. The modified encryption function of
exponential ElGamal looks as follows:
\begin{equation}
\hat{Enc_y} : \mathbb{Z}_q \times \mathbb{Z}_q \rightarrow G_q \times G_q,
(a, b)=\hat{Enc_y}(\hat{m}, r)=(g^r, y^r \cdot \hat{g}^{\hat{m}}) \notag
\end{equation}
In order to decrypt a given ciphertext, we apply the ordinary ElGamal decryption
function and obtain the value $\hat{g}^{\hat{m}} \in G_q$. To reveal $\hat{m}$,
we would have to compute the discrete logarithm, which is a hard problem in
general. If we have knowledge about all the possible plaintexts (e.g. we know
that a ballot either contains 0 for \emph{no} or 1 for \emph{yes}), we can just try all
possible plaintexts. 

Exponential ElGamal has another property which is important in the context of
e-voting: It transforms ElGamal from a multiplicative homomorphic function into
an additive homomorphic function. Homomorphic encrpytion is discussed in detail in
Section \ref{sec:homenc}.

\subsection{Homomorphic encryption}
\label{sec:homenc}
The ElGamal cryptosystem has a property
which is important for building verifiable e-voting schemes, namely the property
of \emph{homomorphism}. Given two mathematical groups $(X,\oplus)$ and
$(Y,\otimes)$, a mathematical function $f:X \rightarrow Y$ is $(\oplus, \otimes)$-homomorphic if
the following condition holds:
\begin{equation}
f(m_1) \otimes f(m_2) = f(m_1 \oplus m_2) \notag
\end{equation}
The ElGamal function offers exactly that property for its encryption function
$Enc_y:G_q \times \mathbb{Z}_q \rightarrow G_q \times G_q$ using the same public key $y$:
\begin{equation}
Enc_y(m_1, r_1) \cdot Enc_y(m_2, r_2) = Enc_y(m_1 \cdot m_2, r_1 + r_2)
\notag
\end{equation}
This property allows to calculate the encrypted product of all messages out of
the encrypted messages and decrypt only the result. Applied to e-voting, the
result can be computed out of the encrypted ballots and only the final result
needs to be decrypted. The ballots themselves can remain encrypted, which is
important for maintaining privacy. The ElGamal cryptosystem is homomorphic
respective to the multiplication operation in the message part, which is not
very fortunate for counting votes. Votes should be summed up in order to get the
final result. In order to reach this, we use the following mathematical property:
\begin{equation}
x^a \cdot x^b = x^{a+b} \notag
\end{equation}

This is exactly the property that we obtain by using exponential ElGamal as
discussed in \Vref{sec:expelgamal}. The product of all ciphertext
values results in the encrypted sum of the cast ballots, encrypted using the
exponential ElGamal encryption function $\hat{Enc_y}:\mathbb{Z}_q \times
\mathbb{Z}_q \rightarrow G_q \times G_q$:

\begin{equation}
\hat{Enc_y}(m_1, r_1) \cdot \hat{Enc_y}(m_2, r_2) = \hat{Enc_y}(m_1 + m_2, r_1 +
r_2)
\notag
\end{equation}

In order to obtain the sum of $m_1$ and $m_2$, the computation of a discrete
logarithm is required, which is considered as a hard problem. In the scenario of
e-voting, we know what the possible solutions are. The number of possible
solution is merely the number of ballots which have been cast, so we can just
iterate through all the possible solutions and stop as soon as it matches with
the decrypted product.


\subsection{Secret sharing}
\label{sec:secretsharing}
In e-voting scenarios it is crucial that not a
single entity can manipulate the result or reveal single votes. This
responsibility, or in our case the secret key which is needed to obtain the
final result, has to be spread across a set of trustees. In the CGS97 scheme,
this property is achieved by using a secret sharing mechanism as proposed by Adi
Shamir in 1979 \cite{Shamir79}. This scheme even allows to define a so called
\textit{threshold}, which defines the minimal amount of participating trustees
in order to decrypt the result. Such a system is also known as a
$(t-n)$-threshold scheme, where $n$ defines the number of shares which are
issued at the beginning and $t$ defines the number of participants needed to
recover the secret. In a first step, a trusted dealer defines a polynomial
function $f(x)$ with degree $t-1$ and random coefficients. Each trustee $A_{i |
1 \leq i \leq n}$ gets its share $s_i=f(i)$. Secret itself, at this stage only
known by the trusted dealer, is defined by $s=f(0)$. In order to reproduce the
secret using the shares distributed among all trustees, we can interpolate the
shares in order to reproduce the coefficients and therefore also $f(x)$. Since a
polynomial function of degree $t-1$ needs at least $t$ points to reproduce using
an interpolation technique such as Lagrange interpolation, the secret can only
be reproduced if at least $t$ trustees are collaborating. 

The approach of Shamir is quite simple, has one major drawback though: It
requires a trusted authority which has the knowledge of the shared key. It it
would be nice to have a scheme where the group of trustees collaborate to create
the shares in a way that nobody can derive the private key unless a sufficient
amount of trustees collaborate. In 1991, Torben Pryds Pedersen proposed a scheme
where this trusted dealer is no longer needed \cite{PED91}. In this scheme, all the
trustees perform the steps of the trusted dealer in the scheme of Shamir, namely
they generate a polynomial function with random coefficients of degree $t-1$ and
calculate a point for each trustee in the system. These points are then
\textit{secretly} communicated individually to each trustee.

\subsection{Zero-knowledge proofs}
\label{sec:zeroknowledgeproofs}
Zero-knowledge proofs (ZKPs) in their sense are conversations between a
\textit{prover} and a \textit{verifier}. They allow the prover to demonstrate,
that she knows a secret without actually revealing the secret itself. The
conversation is similar to a challenge-response protocol, the verifier asks the
prover certain questions about the secret and the verifier answers them. This
kind of conversations are also known as $\Sigma$-protocols. Of course, the
prover could just guess the correct answer to the question and cheat, but if the
verifier repeats the challenge process with a different input, chances are
almost zero that the prover can guess all the correct answers if he is not in
fact in possession of the secret. Zero-knowledge proofs offer therefore a
\textit{probabilistic} security. In the context of e-voting, these Zero
Knowledge Proofs are used to make sure that none of the participants are
cheating.

\subsubsection{Non-interactive zero-knowledge proofs}
Zero-knowledge proofs as described above are \textit{interactive} conversations
between a prover and a verifier. This also means that the prover proofs only to
the verifier that she has knowledge of the secret. Of course, any observer could
observe the conversation, but there is no way to determine whether the verifier
actually accepts the the proof or not. The verifier could of course testify that
the prover has the knowledge of the secret, but that would require a trust
relationship between the observer and the verifier. In an e-voting scenario, we
need a proof which can be verified by anybody and doesn't require an interactive
conversation between the verifier and the prover. Such proofs are called
\textit{Non-Interactive zero-knowledge proofs} or \textit{NIZKPs}. The
foundation of these NIZKPs were introduced by Amos Fiat and Adi Shamir in 1986
\cite{FS87}, later known as the \textit{Fiat-Shamir heuristic}. Instead of the
verifer challenging the prover, the prover challenges himself by using a
\emph{hashfunction} such as SHA-256. The result of such a non-interactive
zero-knowledge Proof is similar to a digital signature. Once published,
everybody can verify the integrity of the data over which the signature has been
calculated. In a similar way, non-interactive zero-knowledge proofs can be
verified, with the important difference that the secret of course remains
secret. So far, we didn't specify what a secret actually is. There are multiple
types of secrets and therefore also multiple types of zero-knowledge proofs, but
due to the work of Ueli Maurer \cite{Maurer09} we can formulate a general
recipe how such a NIZKP is assembled:

Let $(X,\oplus)$ and $(Y,\otimes)$ be two mathematical groups and
$f:X \rightarrow Y$ be a one-way homomorphic function. The prover wants to prove
that she knows the preimage $\alpha \in X$ of the publicly known value $\beta \in
Y$, where $\beta=f(\alpha)$. In order to prove the knowledge of the value
$\alpha$, the prover performs the following steps:
\begin{enumerate}
  \item Choose a uniformly random value $\omega \in_R X$
  \item Compute $t=f(\omega)$
  \item Compute $c=H(\beta||t)$, where $H$ represents a \emph{hashfunction} such
  as SHA-256
  \item Compute $s=\omega \oplus c \cdot \alpha$
  \item Publish the proof $\pi = (t,s)$
\end{enumerate}

A verifier can now calculate $c=H(\beta||t)$ and check whether the following
condition holds:
\begin{equation}
	f(s) \stackrel{?}{=} t \otimes \beta^c \notag
\end{equation}

 In the following paragraphs, the different types of proofs are briefly
 explained and we also make the adaption of the schema above.
 
\subsubsection{Proving the Knowledge of Discrete Logarithm}
This type of Proof was first presented by Claus Peter Schnorr in 1991
\cite{Schnorr91}. We have seen that the ElGamal encryption function has the
homomorphic property \myref{sec:homenc}. Therefore, it is possible to prove the
knowledge of the plaintext of a given ciphertext by applying the scheme above.
We define the ElGamal encryption function using the public key $y$ and the
predefined generator $g$ of the group $G_q$ as follows:
\begin{equation}
	Enc_y:G_q \times \mathbb{Z}_q \rightarrow G_q \times G_q, (a,b) = Enc_y(m,
	r)=(g^r, y^r \cdot m) \notag
\end{equation} 

Since $g$ and $y$ are publicly known values, the knowledge of the value $r$
implies the knowledge of $m$, therefore the prover only needs to prove the
knowledge of $r$, which can be done by substituting the following variables in
the generic scheme above:
\begin{align}
  X &= \mathbb{Z}_q \notag \\
  Y &= G_q \notag \\
  \alpha &=r \notag \\
  \beta &=r \notag \\
  f(x) &= g^x \notag
\end{align} 


This translates to the following steps:

\begin{enumerate}
  \item Choose a uniformly random value $\omega \in_R \mathbb{Z}_q$
  \item Compute $t=g^\omega$
  \item Compute $c=H(a||t)$, where $H$ represents a \emph{hashfunction} such
  as SHA-256
  \item Compute $s=\omega + c \cdot r$
  \item Publish the proof $\pi = (t,s)$
\end{enumerate}

A verifier can now verify the knowledge of $r$ and therefore $m$ by calculating
$c=H(a||t)$ and check whether the following condition holds:
\begin{equation}
	g^s \stackrel{?}{=} t \cdot a^c \notag
\end{equation}

\subsubsection{Proving the Equality of Discrete Logarithms}
This type of proof due to David Chaum and Torben Pryds Pedersen \cite{CP93}
that proofs the relation $log_{g_1} c_1 = log_{g_2} c_2 $ for two values $c_1 =
g_1^m$ and $c_2 = g_2^m$, where $g_1$ and $g_2$ are generators of the
mathematical group $G_q$. In order to prove this relation, we can again
substitute the variables in the scheme above:
\begin{align}
  X &= \mathbb{Z}_q \notag \\
  Y &= G_q \times G_q \notag \\
  \alpha &=m \notag \\
  \beta &=(c_1, c_2) \notag \\
  f(x) &= (g_1^x, g_2^x) \notag
\end{align} 

This translates to the following steps:

\begin{enumerate}
  \item Choose a uniformly random value $\omega \in_R \mathbb{Z}_q$
  \item Compute $t=(g_1^\omega, g_2^\omega)$
  \item Compute $c=H(c_1||c_2||t)$, where $H$ represents a
  \emph{hashfunction} such as SHA-256
  \item Compute $s=\omega + c \cdot m$
  \item Publish the proof $\pi = (t,s)$
\end{enumerate}

A verifier can now verify the relation by calculating
$c=H(c_1||c_2||t)$ and check whether the following condition holds:
\begin{equation}
	(g_1^s, g_2^s) \stackrel{?}{=} t \cdot (c_1, c_2)^c \notag
\end{equation}


\subsubsection{Proving Validity}
Proofs of validity are used to prove that a certain image of a homomorphic one
way function is in fact the image of a preimage and the preimage is an element
of a set of possible preimages. The proof however does not reveal which preimage
it actually is. This type of prove is not a straight forward application of the
generalzed schema of Maurer \cite{Maurer09}, it is rather an OR-combination
knowledge proofs. The idea is to \emph{simulate} accepting conversations for the
preimages which do not correspond to the calculated image and combine those
simulated proofs with the actual proof from the image we calculated. 

Let $(X,\oplus)$ and $(Y,\otimes)$ be two mathematical groups, $f:X \rightarrow
Y$ be a one-way homomorphic function and $A=\{\alpha_1,\ldots,\alpha_n\} \subseteq X$
be a set of $n$ possible preimages. The prover wants to prove that the publicly known
image $\beta=f(\alpha_i) \in Y$ belongs to a preimage $\alpha_i$ without
revealing which preimage it actually is. 

\paragraph{Simulation.} In this phase, we are simulating an accepting
conversation for all the possible preimages without the the preimage for which
the proof is being created. This can be done by executing the following step:

\begin{enumerate}
  \item Compute $\beta_j = f(\alpha_i \otimes \alpha_j^{-1})$ for $j=(1, \ldots,
  i-1, i+1, \ldots, n)$
\end{enumerate}
 
\paragraph{Proof Generation.} Now the proof can be created using the
values $b_j$ created in the simulation phase, the chosen preimage $a_i$ and the
index $i$.
 
\begin{enumerate}  
  \item Select challenges $(c_1, \ldots, c_{i-1}, c_{i+1}, \ldots, c_n) \in_R
  X^{n-1}$
  \item Select responses $(s_1, \ldots, s_{i-1}, s_{i+1}, \ldots, s_n) \in_R
  X^{n-1}$
  \item Compute commitments $t_j=f(s_j) \cdot \beta_j^{-c_j} \in Y$ for
  $j=(1, \ldots, i-1, i+1, \ldots, n)$
  \item Select $\omega_i \in_R X$
  \item Compute $t_i=f(\omega_i) \in Y$
  \item Compute challenge $c\in X$ with hashfunction\footnote{Usually a
  hashfunction such as SHA-256 is used, which does not necessary produce an
  element of $X$. Some sort of mapping is required.} $H$:
  $c=H(\beta_1||\ldots||\beta_n||t_1||\ldots||t_n)$
  \item Compute $c_i=c\oplus(\sum_{j=1, j \neq i}^n c_j)^{-1} \in X$
  \item Compute $s_i=\omega_i \oplus c_i \cdot \alpha \in X$
  \item Publish the proof $\pi = (t_1,\ldots,t_n,c_1,\ldots,c_n,s_1,\ldots,s_n)$
\end{enumerate}

\paragraph{Verification.} A verifier can verify a published proof $\pi$ by
verifying that the following conditions hold:
\begin{align}
	f(s_j) &\stackrel{?}{=} t_j \cdot \beta_j^{c_j} \text{ for } j=(1...n)
	\notag \\
	 \sum_{j=1,}^n c_j &\stackrel{?}{=}
	 H(\beta_1||\ldots||\beta_n||t_1||\ldots||t_n)
	 \notag
\end{align}
The formal description above depicts how a general OR-proof is assembled. The
cryptographic building block we need for our implementation is a proof of
validity, which proves that the plaintext $m$ of a certain ElGamal ciphertext
$(a, b)$ is an element of a set of possible plaintexts $M=\{m_1, \ldots, m_n\}$. This
scenario can be seen as an instance of an OR-proof. We can show this relation by
substituting the variables in the generalized schema above as follows:
\begin{align}
  X &= \mathbb{Z}_q \notag \\
  Y &= G_q \times G_q \notag \\
  \alpha &=r \notag \\
  \beta &=(a, b) \notag \\
  f(x) &= (g^x, y^x) \notag
\end{align} 
Note that as a function $f(x)$, we use the \emph{identity function} of ElGamal,
which encrypts the identity element of the domain (in the case of the
multiplicative cyclic group $G_q$, the identity element is $1$).

\subsection{Bulletin Board}
\label{sec:bulletinboard}
The so called bulletin board is the public communication channel which is used
to communicate between the participants of the election or poll. It is a
transcript of all the communication steps between the participants and therefore
contains encrypted ballots, zero knowledge proofs, etc. The bulletin board is
also available for observers. Using the content of the bulletin board, everybody
can verify that all the participants are following the protocol or that
dishonest participants are excluded from the voting process. A voter can verify
that his/her own ballot is counted properly and also reflects in the final
result. In theory, it is not possible to delete anything from a bulletin board
(append only). There are several approaches on how to create such an
append-only bulletin board. The CGS97 protocol itself only assumes this
append-only property. Append-only bulletin boards are discussed in more detail
in \cite{HL09}. Since the bulletin board is a good target for a
denial-of-service attack, it is a good idea to replicate the content of the bulletin board to multiple systems.


\section{The Voting Scheme CGS97}
\label{sec:CGS97}
In 1997, Ronald Cramer, Rosario Gennaro and Berry Schoenmakers proposed a scheme
\cite{CGS97} which allows to do e-voting in a secure and verifiable manner. The participants of
the protocol can be divided into four different roles:
\begin{itemize}
  \item \textbf{Administrator:} The administrator is responsible for setting up
  the election by defining the question and the possible options, the
  electorate and the voting period. The administrator is also responsible for
  orchestrating the activities during a voting cycle.
  \item \textbf{Voter:} A voter is somebody who is eligible to participate on
  an election or a poll. 
  \item \textbf{Trustee:} A trustee is somebody who helps setting up the
  election by creating an asymmetric key pair in cooperation with other
  trustees.
  At the end of the election phase, the trustees have to cooperate in order to
  reveal the result of the election or the poll.
  \item \textbf{Observer:} An observer is somebody who wants to verify that all
  the participants of an election or a poll behave as they are supposed to.
\end{itemize}

In the scenario of an ad-hoc voting system as we are going to develop during
this project, the voter, trustee and observer roles can be combined and all
participant impersonate these roles. The administrator role needs to be assigned
to one particular participant.

\subsection{Security Properties}
\label{sec:secproperties}
In \Vref{sec:evoting} we discussed some security requirements which
are desirable for an e-voting system. The following paragraph assesses the CGS97
protocol against these requirements.

\paragraph{Democracy.} The CGS97 protocol doesn't specify how the access to
the virtual voting both is controlled. The e-voting system which implements the
CGS97 protocol needs a sufficiently secure authentication mechanism in order to
fulfill the democracy requirement. The criteria that only one valid vote can be
cast is also tied to the authentication mechanism.

\paragraph{Accuracy.} A zero-knowledge proof is used to prove that all valid
ballots are tallied correctly. Since the bulletin board is publicly available,
this can be verified by anybody.

\paragraph{Universal verifiability.} The bulletin board which is publicly
available (for voters as well as for observers who don't participate at the
election or poll) assures verifiability for anybody. The bulletin board can also
be seen as a transcript of the conversation between the actors during a voting
cycle. 

\paragraph{Individual verifiability.} The voter can identify her own vote on
the bulletin board along with the other votes. By verifying the zero-knowledge
proofs of the trustees which have been created during the tallying and
decryption of the result, the voter can be sure that her vote has been included
in the final result.

\paragraph{Privacy.} The ballot which has been encrypted and cast by the
voter always remains encrypted unless a sufficient amount of trustees decide
conspire and decrypt single votes and reveal information on how the participants
voted.

\paragraph{Receipt-freeness.} This property is one of the weak points of the
CGS97 voting scheme. A voter is able to derive a proof which shows how she
voted.

\paragraph{Coercion-resistance.} Coercion resistant protocols have mechanisms
to allow the voter to lie about the vote when under coercion. The voter pretends
to vote according to the wish of the coercer while voting in fact according to
her intention. Unfortunately, CGS97 does not offer this possibility.

\paragraph{Robustness.} The CGS97 offers robustness by including a
threshold secret sharing mechanism that assures a working system as long as a
certain number of properly behaving trustees remain in the system. The other
crucial part which requires robustness is the bulletin board, which should be
replicated for achieving robustness. The CGS97 scheme doesn't specify in detail
how this should be achieved. A possible approach can be found in \cite{HL09}.

\subsection{Usability Properties}
\label{sec:usabilityproperties}
It is a generally know fact that high security measure have a negative impact on
the usability of a certain application. In e-voting, usability is a crucial
factor since the potential average voter is not expected to have a high
expertise in cryptography and information security. In the CGS97 scheme, casting
a vote is surprisingly easy and also the computation costs are fairly low,
regardless of how many trustees the election has or how big the electorate is.

A similar property is also valid for the trustees, the complexity of verifying
votes, tallying and decrypting always remains linear which is a huge advantage
compared to other e-voting schemes.

\subsection{The Protocol}
The CGS97 voting scheme has a well defined procedure on how an voting should be
performed. This procedure is explained in this section. Formally we define the set
of $n$ trustees as $T=\{\tau_1, \ldots, \tau_n\}$ and the set of $l$
voters as $V=\{v_1, \ldots, v_l\}$. Furthermore, the role of an
\emph{administrator} $A$ needs to be assigned to a participant. The process of these role assignments
strongly depends on the political structure of the organization which runs
this election and is considered as an administrative task and is hence not
further discussed at this point. In order to hold an election, all the involved
parties collaborate as follows:

\paragraph{Initialization Phase.}
All the parties need to agree on the ElGamal parameters $p, q,$ and $g$. These
parameters can be used as fixed parameters and do not need to be redefined for
each election. The details can be found in \Vref{sec:elgamal}. One
particular participant of the election, let's define it as the
\emph{administrator}, defines the question and all the possible options of the
upcoming election.

\paragraph{Key Generation Phase.}
All the trustees execute a robust threshold key generation protocol as discussed
in \Vref{sec:thresholdsecretsharing}. At this stage, a \emph{threshold}
$t\leq|T|$ must be defined. This parameter defines the minimal number of
trustees which are required to decrypt the final result in the tallying phase.
The transcript of the key generation protocol is posted on the bulletin board
\myref{sec:bulletinboard}. The outcome of this key generation protocol is an
ElGamal public key $y$ which needs to be communicated to all the voters $V$.

\paragraph{Voting Phase.}
During a well defined time window, the voters can now create and cast a ballot.
To do so, the voter encrypts his ballot using the public key $y$ and posts his
ballot to the bulletin board. Note that we need to use \emph{exponential
ElGamal} \myref{sec:expelgamal} so that we can \emph{add} the result in an upcoming
stage.
Futhermore, the voter needs to post a proof of validity \myref{sec:proofofvalidity} on the
bulletin board in order to prove that encrypted ballot indeed contains a valid option of the election.

\paragraph{Tallying Phase.}
The tallying phase can start as soon as the voting period has come to an end. In
order to tally an election, the ciphertexts of the valid ballots are multiplied
and posted to the bulletin board. Since all the encrypted ballots are publicly
available, this multiplication step can be verified easily by doing the
multiplication individually and compare the result to the values on the
bulletin board. The product of all the valid ballots now represents the
\emph{encrypted result} of the election, which has to be decrypted by the
trustees. In order to do so, the trustees execute a threshold decryption
protocol in order to reveal the result of the election. Note that we only need a
subset $\Lambda \subseteq T$ with minimal order $t$ in order to execute the
protocol properly. The transcript of the protocol and the result are posted on
the bulletin board. Furthermore, each participating trustee has to provide a
proof that he performed his decryption step in a correct fashion.

\section{UniCrypt}
\label{sec:unicrypt}
UniCrypt is the name of a cryptographic library which has been developed by the
E-Voting Group of the Bern University of Applied Sciences. The main goal of this
project was to create a platform on which upcoming projects can be built on.
UniCrypt tries to look at cryptographic functions such as cryptosystems,
signature systems, hash functions etc. from a mathematical point of view. Cipher
text, plain text, signatures, etc. are treated as elements of mathematical
groups, the cryptographic functions are treated as mathematical functions. This
is required because in secure protocols as we use in the area of e-voting, we
depend on the plain textbook implementation of the cryptographic functions
without functions like automatic padding, encoding, etc. This is the main
difference between UniCrypt and other cryptographic libraries.

At the moment, UniCrypt has only been used in classic Java environments, but not
on mobile devices. Further information about UniCrypt can be found in
\cite{ritter12}.

\section{InstaCircle}
\label{sec:instacircle}
InstaCircle is the name of a project that has been implemented as a preparation
of this master thesis. It is intended as a platform to connect mobile devices
in a confined space to a Wi-Fi ad-hoc network in order to exchange broadcast
and unicast messages. In order to keep the network traffic as low as possible,
the basic communication relies on broadcast messages. Since it is not possible
implement a reliable protocol using broadcast techniques, mechanisms to
compensate message losses have been implemented. These resending mechanisms rely
on a reliable unicast channel.

From a usability perspective, the process on agreeing to a communication channel
for an ad-hoc network is quite challenging. All the participants need to have
the know-how to switch their devices to the correct Wi-Fi network and enter the
correct keys and passwords. InstaCircle tries to improve the usability by
allowing the users to exchange the configuration by sharing a QR code or a NFC
tag. Once a user has set up a conversation, other participants can join the
conversation by reading the QR code or the NFC tag shared by the initiator of
the conversation.

InstaCircle is currently available for Android devices only. Further information
about InstaCircle can be found in \cite{ritter13a}.



\chapter{Organization}
\label{cha:organization}
This project is implemented as a master thesis project as a graduation project
during the Master of Science in Engineering studies (MSE) at the Bern University
of Applied Sciences. This project values 27 ECTS credits and is spread among two
semesters. The hand in of the project is planned for February 7, 2014.

During the same period, a fellow master student is working on a very similar
topic, namely the implementation of the decentralized e-voting system proposed by
Dalia Khader et al. \cite{HKRS12}. This project will also be based on the
InstaCircle communication infrastructure \cite{ritter13a} and the UniCrypt
cryptographic library \cite{ritter12} which have been developed earlier during
the master studies. It is the plan to establish a collaboration between these
two project, mainly for the development of the graphical user interface. A
collaboration between these project also makes it easier to compare the
approaches and evaluate the advantages and disadvantages of the two approaches.

\section{Planning}
\label{sec:planning}
The gantt charts in figure \ref{fig:planningpart1} and figure
\ref{fig:planningpart2} visualize the steps which are planned on the time axis
during the project period. At the first stage, we define the project setup,
later on the cryptographic components will be put in place and will be tested on
the Android platform. At some point a detailed story board for the graphical
user interface will be created. Especially this project will be done in a strong
collaboration with the other project which deals with decentralized e-voting
systems. At this stage, we have prepared all the components we need in order to
implement the actual e-voting system. The part which will consume most of the
time budget will be the implementation and the testing of the actual e-voting
logic. The last part will be dedicated to writing the project report of this
master thesis project.

\begin{sidewaysfigure}[ph]
\noindent\resizebox{\textwidth}{!}{
	\begin{tikzpicture}[x=.5cm, y=1cm]
	\begin{ganttchart}
	[vgrid, hgrid,
	group/.style={draw=black, fill=black!50}, y unit chart=0.8cm]{26}
	\gantttitle{\textbf{Planning Master Thesis (part 1)}}{26} \\
	\gantttitlelist{8,...,33}{1} \\
	\ganttmilestone{Kickoff}{0.5} \\
	\ganttgroup{Writing of Project Proposal}{1}{6} \\
	\ganttgroup{Secret Sharing scheme}{6}{10} \\
	\ganttbar{Analyzing the possibilities of threshold secret sharing}{6}{8} \\
	\ganttbar{Implementing secret sharing into UniCrypt}{7}{10} \\
	\ganttbar{Testing the secret sharing scheme}{9}{10} \\
	\ganttgroup{Proof of Validity}{11}{13} \\
	\ganttbar{Implementing proof of validity into UniCrypt}{11}{12} \\
	\ganttbar{Testing of the proof of validity implementation}{12}{13} \\
	\ganttgroup{Evaluation of UniCrypt on Android}{14}{15} \\
	\ganttbar{Testing out Example code on Android}{14}{14} \\
	\ganttbar{Integration of code into Android project}{15}{15} \\
	\ganttmilestone{Building blocks ready}{15.5} \\
	\ganttgroup{Implementation of e-voting Application of Android}{20}{26} \\
	\ganttbar{Design of storyboard}{20}{24}\\
	\ganttbar{Implementation of the User Interface}{23}{26}\\
	\ganttbar[bar height=0.0, bar/.style={draw=none}]{Implementation of the
	e-voting logic}{0}{0}\\
	\ganttgroup[group height=0.0, group/.style={draw=none}]{Testing}{0}{0} \\
	\ganttgroup{Documentation / Report}{1}{8} \\
	\ganttmilestone[milestone height=0.0, milestone/.style={draw=none}]{Hand in}{0}	
	\end{ganttchart}
	\end{tikzpicture}
}
\caption{Planning part 1}
\label{fig:planningpart1}
\end{sidewaysfigure}
	

\begin{sidewaysfigure}[ph]
\noindent\resizebox{\textwidth}{!}{
	\begin{tikzpicture}[x=.5cm, y=1cm]
	\begin{ganttchart}
	[vgrid, hgrid,
	group/.style={draw=black, fill=black!50}, y unit chart=0.8cm]{26}
	\gantttitle{\textbf{Planning Master Thesis (part 2)}}{26} \\
	\gantttitlelist{34,35,36,37,38,39,40,41,42,43,44,45,46,47,48,49,50,51,52,1,2,3,4,5,6,7}{1}
	\\
	\ganttmilestone[milestone height=0.0,
	milestone/.style={draw=none}]{Kickoff}{0.5} \\
	\ganttgroup[group height=0.0, group/.style={draw=none}]{Writing of Project
	Proposal}{}{} \\
	\ganttgroup[group height=0.0, group/.style={draw=none}]{Secret Sharing
	scheme}{0}{0} \\
	\ganttbar[bar height=0.0, bar/.style={draw=none}]{Analyzing the possibilities
	of threshold secret sharing}{0}{0} \\
	\ganttbar[bar height=0.0, bar/.style={draw=none}]{Implementing secret sharing
	into UniCrypt}{0}{0}
	\\
	\ganttbar[bar height=0.0, bar/.style={draw=none}]{Testing the secret sharing
	scheme}{0}{0} \\
	\ganttgroup[group height=0.0, group/.style={draw=none}]{Proof of
	Validity}{0}{0} \\
	\ganttbar[bar height=0.0, bar/.style={draw=none}]{Implementing proof of
	validity into UniCrypt}{0}{0} \\
	\ganttbar[bar height=0.0, bar/.style={draw=none}]{Testing of the proof of
	validity implementation}{0}{0} \\
	\ganttgroup[group height=0.0, group/.style={draw=none}]{Evaluation of UniCrypt
	on Android}{0}{0} \\
	\ganttbar[bar height=0.0, bar/.style={draw=none}]{Testing out Example code on
	Android}{0}{0} \\
	\ganttbar[bar height=0.0, bar/.style={draw=none}]{Integration of code into
	Android project}{0}{0} \\
	\ganttmilestone[milestone height=0.0,
	milestone/.style={draw=none}]{Building blocks ready}{0.5} \\
	\ganttgroup{Implementation of
	e-voting Application of Android}{1}{13} \\
	\ganttbar[bar height=0.0, bar/.style={draw=none}]{Design of
	storyboard}{0}{0}\\
	\ganttbar[bar height=0.0, bar/.style={draw=none}]{Implementation of the User
	Interface}{0}{0}\\
	\ganttbar{Implementation of the e-voting logic}{1}{13}\\
	\ganttgroup{Testing}{7}{19} \\
	\ganttgroup{Documentation / Report}{14}{24.5} \\
	\ganttmilestone{Hand in}{24.5}
	\end{ganttchart}
	\end{tikzpicture}
}
\caption{Planning part 2}
\label{fig:planningpart2}
\end{sidewaysfigure}

\section{Milestones}
\label{sec:milestones}
In order to track the progress of the project, we define some milestones
throughout the project period.
\begin{itemize}
  \item \textbf{Kickoff:} The project is set up and the goals of the project
  are clear. All involved persons know their role and their responsibilities.
  \item \textbf{Building blocks ready:} At this stage, all the components we
  need (e.g. UniCrypt) are prepared and tested on the Android platform.
  \item \textbf{Hand in:} The implementation and the documentation are finished
  and the work is handed over to the advisor.
\end{itemize}

\chapter{Results}
\label{cha:results}
This chapter will be written in a later phase of the project.

\chapter{Discussion}
\label{cha:discussion}
This chapter will be written in a later phase of the project.

\chapter{Conclusion}
\label{cha:conclusion}
This chapter will be written in a later phase of the project.

\printbibliography

\end{document}
